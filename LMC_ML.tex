\documentclass[a4paper, 15pt,usenatbib]{article}
\usepackage[utf8]{inputenc}
\usepackage{amsmath}
\usepackage{amsfonts}
\usepackage{amssymb}
\usepackage{graphicx}
\usepackage{hyperref}
\usepackage{geometry}
\usepackage{fancyhdr}
\usepackage{caption}
\usepackage{subcaption}
\usepackage{natbib}
\usepackage{txfonts}
\usepackage[T1]{fontenc}
\usepackage{ae,aecompl}
\usepackage{multirow}
\usepackage{longtable}
\usepackage{booktabs}

% Hyperref settings to show links in blue without boxes
\hypersetup{
    colorlinks=true,
    linkcolor=blue,
    filecolor=blue,
    citecolor=blue,
    urlcolor=blue,
    linkbordercolor={0 0 1},
    citebordercolor={0 0 1},
    urlbordercolor={0 0 1}
}


\geometry{margin=1in}


\newcommand\apj{Astrophysical Journal}
\newcommand\mnras{Monthly Notices of the Royal Astronomical Society}
\newcommand\prl{Physical Review Letters}
\newcommand\aap{Astronomy and Astrophysics}
\newcommand\physrep{Physics Reports}
\newcommand\araa{Annual Reviews of Astronomy and Astrophysics}
\newcommand\nat{Nature}
\newcommand\aj{The Astronomical Journal}
\newcommand\apjs{The Astrophysical Journal Supplement}
\newcommand\apjl{Astrophysical Journal}
\newcommand\pasa{Publications of the Astronomical Society of Australia}
\newcommand\pasp{Publications of the Astronomical Society of the Pacific}
\newcommand\jcap{Journal of Cosmology and Astroparticle Physics}
\newcommand\aapr{Astronomy and Astrophysics Review}
\newcommand\prd{Physical Review D}
\newcommand\aaps{Astronomy and Astrophysics Supplement Series}
\newcommand\pasj{Publications of the Astronomical Society of Japan}
\newcommand\amp{Annales d'Astrophysique}
\newcommand\fcp{Fundamentals of Cosmic Physics}
\newcommand{\msun}{\mbox{$\,{\rm M}_\odot$}}

% Header and Footer
\pagestyle{fancy}
\fancyhf{}
\fancyhead[L]{\leftmark}
\fancyfoot[C]{\thepage}

% Title Page
\title{Machine Learning Classification of Infrared
Sources in the Large Magellanic Cloud}
\author{Hamidreza Mahani \\
    \\
        Institute for Research in Fundamental Sciences (IPM) \\
        School of Astronomy}
\date{\today}

\begin{document}

\maketitle

\begin{abstract}
    In this project, we aim to classify the point sources in the Large Magellanic Cloud, observed by the Spitzer Space Telescope and presented in the SAGE catalog, using machine learning models. Following the classification, we will study the life cycle of dust for those objects that produce and consume dust. Codes, plots, and other materials related to this project are available in the \href{https://github.com/hmahani/LMC_Machine_Learning}{GitHub repository} for this project.
\end{abstract}

\tableofcontents
\newpage

\section{Introduction}
Provide an introduction to your research. Explain the background, context, and the main objectives of your study. 

\section{Catalogs}
The primary catalog studied in this project is the SAGE catalog, which initially contained 7,048,620 rows and 111 columns. This catalog was cross-matched with the \href{https://vizier.cds.unistra.fr/viz-bin/VizieR-3?-source=I/350&-out.max=50&-out.form=HTML%20Table&-out.add=_r&-out.add=_RAJ,_DEJ&-sort=_r&-oc.form=sexa}{Gaia} \citep{GaiaDR3} and \href{https://vizier.cds.unistra.fr/viz-bin/VizieR-3?-source=II/375&-out.max=50&-out.form=HTML%20Table&-out.add=_r&-out.add=_RAJ,_DEJ&-sort=_r&-oc.form=sexa}{VISTA} \citep{Cioni11_Vista} catalogs, resulting in the final catalog having 7,048,620 rows and 205 columns.

\section{Methodology}
The classes taken into account for the classification of point sources include:

\begin{enumerate}
  \item OB stars
  \item Main sequence stars
  \item A-G super giants
  \item HII regions
  \item Foreground stars
  \item Background galaxies
  \item WR stars
  \item Red super giants
  \item O-AGBs
  \item C-AGBs
  \item LBVs
  \item Post-AGBs
  \item Planetary nebulae
  \item Young stellar objects. 
\end{enumerate}  

Supervised machine learning models will be employed for point source classification. To facilitate this, it is imperative to utilize sources whose class is definitively confirmed as test and training catalogs. The primary choice for confirmed sources is the \citet{Jones17} catalog.

\section{Results}
Present the results of your research. Use tables, figures, and graphs to illustrate your findings. Make sure to label and caption all figures and tables clearly.

\section{Discussion}
Interpret and discuss your results. Explain their significance and how they relate to your research questions and hypotheses. Compare your findings with those of other studies.

\section{Conclusion}
Summarize the main findings of your research. Discuss the implications of your work and suggest possible directions for future research.

\section{Acknowledgements}
Acknowledge any individuals or organizations that assisted with your research. This may include funding sources, colleagues, or mentors.

\section{References}
%\bibliographystyle{plain}
%\bibliography{LMC}

%%%%%%%%%%%%%%%%%%%% REFERENCES %%%%%%%%%%%%%%%%%%

% The best way to enter references is to use BibTeX:

\bibliographystyle{mnras}
\bibliography{LMC} % if your bibtex file is called example.bib


%%%%%%%%%%%%%%%%% APPENDICES %%%%%%%%%%%%%%%%%%%%%

\end{document}
